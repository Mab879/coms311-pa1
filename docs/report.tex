\documentclass[10pt,letterpaper]{article}
\usepackage[utf8]{inputenc}
\usepackage{amsmath}
\usepackage{amsfonts}
\usepackage{amssymb}
\usepackage[left=2cm,right=2cm,top=2cm,bottom=2cm]{geometry}
\author{Matthew Burket and Joel May}
\title{Programming Assignment 1 Report}
\begin{document}
\maketitle
\section{Hash Table}
\subsection{Pseudo Code} 
\subsubsection{\texttt{add(Tuple t)}}
\begin{verbatim}
add(Tuple t)
   hash = hashFunction.hash(t.key)
    if table[hash] == null
        table[hash] = new LinkedList();
    table[hash].add(t)
    if (loadFactor() > 0.7) 
        newSize = getPrime(size() * 2)
        newTable = LinkedList[newSize]
        newHashFunction = new hashFunction(newSize)
        for list in table
            for tuple in list
                hash =  newHashFunction.hash(t.key)
                newTable[hash].add(t)
\end{verbatim}
\subsubsection{\texttt{search(Tuple t)}}
\begin{verbatim}
search(Tuple t) 
    count = 0;
    hash = hashFunction.hash(t.key)
    for Tuple loopTuple in table[hash]
        if (loopTuple.equals(t))
            count++
    return count
\end{verbatim}
\section{BruteForceSimilarity}
\subsection{Pseudo Code}
\subsubsection{\texttt{lengthOfS1()}}
\subsubsection{\texttt{lengthOfS2()}}
\subsubsection{similarity}
\subsection{Data Structures}
The BruteForceSimilarity uses ArrayList as its primary data structure. It doesn't directly use any arrays.
\subsection{Run Time}
\subsection{Similarity Results}
\section{HashStringSimilarity}
\subsection{Pseudo Code}
\subsubsection{\texttt{lengthOfS1()} and \texttt{lengthOfS2()}}
\texttt{s} is the hash table of shingles. \texttt{lengthOfS1} and \texttt{lengthOfS2} will pass the correct HashTable to this method.
\begin{verbatim}
vectorLength(IterableHashTable s)
    HashTable countedValues = HashTable(13)
    sum = 0
    for Tuple i in S
        Tuple hashForCompare = t;
        if (countedValues.search(t) == 0)
            countedValues.add(hashForCompare);
            occurrences = S.search(hashForCompare);
            sum += occurrences * occurrences;
    return Math.sqrt(sum)
\end{verbatim}
\subsubsection{\texttt{similarity()}}
\subsection{Data Structures }
\subsection{Run Time}
\subsection{Similarity Results}
\section{HashCodeSimilarity}
\subsection{Pseudo Code}
\subsubsection{\texttt{lengthOfS1()}}
\subsubsection{\texttt{lengthOfS2()}}
\subsubsection{\texttt{similarity()}}
\subsection{Data Structure }
\subsection{Run Time}
\section{Similarity Results}
\end{document}